\documentclass[10pt]{article}
\usepackage{makeidx}
\usepackage{multirow}
\usepackage{multicol}
\usepackage[dvipsnames,svgnames,table]{xcolor}
\usepackage{graphicx}
\usepackage{epstopdf}
\usepackage{ulem}
\usepackage{hyperref}
\usepackage{amsmath}
\usepackage{amssymb}
\usepackage[utf8]{inputenc}
\author{Juan Francisco Zamora Osorio}
\title{SPONSORING STATEMENT POSTDOCTORADO 2017}
\usepackage[paperwidth=612pt,paperheight=792pt,top=28pt,right=34pt,bottom=35pt,left=34pt]{geometry}

\begin{document}

\noindent \textbf{SPONSORING RESEARCHER STATEMENT:}
%\noindent \textbf{Declaración del Investigador Patrocinante:}

\vspace{15pt}
\noindent Describe your role on this proposal development, including the benefit(s) of incorporating the applicant to your line of research and work group, as well as other aspects contributing to the relevance of its execution. The maximum length for this section is 1 page (Verdana font size 10, letter size is suggested).

%La labor principal del investigador patrocinante será la de coordinar el trabajo de investigación y de aportar sus conocimientos en el área de M\'aquinas de Aprendizaje Distribuidas. El investigador es un académico jornada completa y tiene experiencia investigando en el área en cuesti\'on, demostrable con publicaciones indexadas y conferencias en las áreas de regresi\'on distribuida y métodos de aprendizaje automático. 

The sponsoring researcher collaborated in the elaboration of the work plan and he will coordinate the progress of the project. Additionally, this researcher will contribute with his expertise in the field of Distributed Machine Learning algorithms. The researcher is a full-time professor of the sponsoring institution and has a well known research experience supported by several publications in WOS journals and international conferences in distributed data regression
and neural networks for classification and regression.

%De manera m\'as espec\'ifica el investigador patrocinante realizar\'a las siguientes actividades durante la duraci\'on del proyecto:
More specifically the researcher will perform te following activities during the execution of this project:
\begin{enumerate}
\item %Colaboraci\'on t\'ecnica y en infraestructura de procesamiento de datos para el proyecto, además de ayuda en redacci\'on de reportes.
    Collaboration in technical aspects for the design of the algorithms and in infrastructure for data processing.
\item %Coordinaci\'on de las etapas descritas en el Plan de Trabajo.
    Coordination of the stages described in the work plan.
\item %Supervisi\'on de las metas establecidas en el proyecto.
    Supervision of the goals contemplated in the project.
\item %Estudio y discusi\'on de la literatura relevante.
    Discussion and study of the relevant literature to the project.
\item %Colaboraci\'on en el dise\~no del m\'etodo propuesto.
    Contribution on the discussion and elaboration of the proposed methods and their experimental evaluation.
\item %Elaboraci\'on de art\'iculos ISI en revistas indexadas.
    Participation in the elaboration of written documents for dissemination and publications in WOS articles.
\end{enumerate}

%En el núcleo del área de M\'aquinas de Aprendizaje existen primordialmente 2 problemas, el de clasificación y regresión. Estos 2 problemas comparten muchas similitudes, como la generación y selección de características. De hecho es posible ver de manera muy general el problema de clasificaci\'on como un subproblema del de regresión, pero con variable dependiente categórica. El problema de agrupamiento (\emph{Clustering}) cae en la categoría de clasificación no supervisada.
%En trabajos anteriores del investigador patrocinante ha abordado la tarea de regresión en escenarios distribuidos y también de \textit{BigData}. El tema de investigación propuesto, comparte sin duda las características de los problemas anteriormente investigados, por lo que resulta beneficioso poder explorar el mismo problema pero visto desde la tarea de agrupamiento. 

At the core of Machine Learning there are two main tasks, namely Classification and Regression. Both tasks are tightly related since it is possible to formulate the classification task as a particular case of regression with a categorical target variable. The Clustering task is also considered as an unsupervised classification task. We expect that the experience and knowledge of the sponsoring researcher in Distributed Regression and Classification methods may be exploited in the context of a Distributed Clustering scenario. 
Undoubtedly, the proposed research topic shares the traits of the previously mentioned subjects thus this projects also offers the opportunity to explore the more general Classification problem from a different angle in a distributed data setting. %TODO:revisar esta ultima frase!!

%Para finalizar, en escenarios de BigData se requiere de la participación de especialistas que ataquen problemas complejos desde distintas perspectivas, considerando como herramienta el computo distribuido y paralelo. Como resultado de este proyecto se espera satisfacer esta idea, incorporando nuevos métodos capaces de agrupar colecciones documentales con alta dimensionalidad en escenarios distribuidos, ya sea por el origen natural de los datos o como estrategia de procesamiento.
Finally, BigData problems require the integration of specific knowledge coming from experts that study the phenomenon from varied perspectives that also include techniques for parallel (at the CPU and/or GPU level) and distributed computing. We expect that this project satisfies this restriction by joining this partial efforts (sponsoring and principal researchers) and produces new methods capable of identify meaningful groups of documents from massive and distributed text collections.

\end{document}

\documentclass[10pt]{article}
\usepackage{makeidx}
\usepackage{multirow}
\usepackage{multicol}
\usepackage[dvipsnames,svgnames,table]{xcolor}
\usepackage{graphicx}
\usepackage{epstopdf}
\usepackage{ulem}
\usepackage{hyperref}
\usepackage{amsmath}
\usepackage{amssymb}
\author{Felipe Vásquez Moraga}
\title{SPONSORING STATEMENT POSTDOCTORADO 2017}
\usepackage[paperwidth=612pt,paperheight=792pt,top=28pt,right=34pt,bottom=35pt,left=34pt]{geometry}

\begin{document}

\noindent \textbf{SPONSORING RESEARCHER STATEMENT:}

\vspace{15pt}
\noindent Describe your role on this proposal development, including the benefit(s) of incorporating the applicant to your line of research and work group, as well as other aspects contributing to the relevance of its execution. The maximum length for this section is 1 page (Verdana font size 10, letter size is suggested).


\end{document}
