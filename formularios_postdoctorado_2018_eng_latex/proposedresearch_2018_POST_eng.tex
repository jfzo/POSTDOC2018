\documentclass[10pt]{article}
\usepackage{makeidx}
\usepackage{multirow}
\usepackage{multicol}
\usepackage[dvipsnames,svgnames,table]{xcolor}
\usepackage{graphicx}
\usepackage{epstopdf}
\usepackage{ulem}
\usepackage{hyperref}
\usepackage{amsmath}
\usepackage{amssymb}
\usepackage[square]{natbib}
\author{Felipe Vásquez Moraga}
\title{PROPOSAL ABSTRACT POSTDOCTORADO 2018}
\usepackage[paperwidth=612pt,paperheight=792pt,top=28pt,right=34pt,bottom=35pt,left=34pt]{geometry}


\begin{document}

\noindent \textbf{PROPOSAL ABSTRACT:}

\noindent \textbf{}

{\raggedright
\vspace{3pt} \noindent
\begin{tabular}{|p{514pt}|}
\hline
\parbox{514pt} {\raggedright\vspace{3pt}
The maximum length of this file is \textbf{10 pages} (Must use letter size, Verdana size 10 or similar)\textbf{. }For an adequate evaluation of your proposal merits,this file must include the following aspects: Proposal description, Bibliographical References, Hypothesis, Goals, Methodology, Work Plan, Work in progress (if applicable) and Available Resources.
Make sure to describe the relevance of the proposal topic in relation to the state of the art in the field.
Keep in mind the Bases del Concurso FONDECYT de Postdoctorado 2018 and Application Instructions.} \\
\hline
\end{tabular}
\vspace{2pt}

}

%\section{Proposal}
\section{Introduction}
%\section{Introducción}
%Como consecuencia del crecimiento explosivo de la WEB, la integración de motores de búsqueda como Google en computadoras personales y dispositivos móviles, y del uso de la redes sociales, la tarea de agrupamiento automático de texto, e.g. Tweets o documentos en un motor de búsqueda, resulta cada vez de mayor importancia debido a la necesidad de revelar las categorías subyacentes en grandes volúmenes de datos. La generación de grandes cantidades de documentos sobrepasa hoy en día las capacidades de computadores personales e incluso de equipos de cómputo de alto desempeño. 
As a consequence of the explosive growth of the WEB, the integration of search engines such as Google into personal computers and mobile devices and the wide use of social networks, the task of clustering text data has gained importance due to the necessity of unraveling the underlying categories in large volumes of text data (e.g. Tweets or WEB pages). Nowadays the generation of large amounts of documents surpasses the computational power of personal computers and even of high performance computers.
%A modo de ejemplo, se estima que la cantidad de páginas WEB que motores de búsqueda populares como Google o Yahoo indexan es del orden de las decenas de billón.
As an example, it is estimated that the amount of WEB pages that popular search engines such as Yahoo! and Google index is higher than 40 billion of pages\footnote{\url{http://www.worldwidewebsize.com}}.
%Es por lo tanto de gran interés poder desarrollar técnicas capaces de automáticamente organizar, clasificar y resumir colecciones de documentos que se encuentren distribuidas en múltiples máquinas de una red, y que a su vez hagan uso eficiente del hardware moderno, en particular del paralelismo existente en las arquitectura multi-núcleo (multi-core). En problemas reales como  \textit{Collection Selection} para colecciones distribuídas, en donde para una consulta dada se debe identificar la colección más adecuada dentro de la cual se buscaran documentos relevantes \citep{CM13}, los desafíos relacionados con la escalabilidad y eficiencia en los métodos de \textit{Knowledge Discovery} se han vuelto de gran importancia. Los algoritmos tradicionales operan usualmente con conjuntos de datos cargados completamente en memoria principal. De ahí que puedan realizar operaciones de cálculo de distancias entre pares de documentos de manera muy rápida, visitando cada dato muchas veces. Cuando el tamaño de la colección es mucho mayor a la cantidad de memoria RAM disponible, ya sea por la cantidad de documentos o también por el tamaño de cada uno, este proceso es impracticable debido a restricciones de espacio o capacidad de cómputo.
It is therefore of great interest to develop algorithmic techniques capable of automatically organize, classify and summarize document collections distributed in multiple machines of a network, and that also perform efficiently in modern hardware, particularly within parallel processing frameworks in multi-core architectures. In real problems such as \textit{Collection Selection} for distributed document databases, in which for a given query the computer node containing the most suitable
sub-collection must be selected to answer it,
%~\citep{CM13}
the challenges related to the scalability and efficiency of \textit{Knowledge Discovery} methods have become very important. Traditional algorithms often assume that the whole dataset is loaded into main memory (RAM) and thus every document can be accessed at any time with no access latency. Actual real problems present though challenges since the sizes of the collections explored nowadays are much bigger than the amount of available RAM in modern computers\footnote{Amount of primary memory in modern computers reaches about 128 Gigabytes but a medium size document collection such as RCV1 can reach near 1 Terabyte (depending on the followed text processing procedure).}, either because of the number of documents or the size of the computational representation of the documents.
%, this ideal scenarioloading process is unfeasible due to real constraints in the storage or computational capabilities.

%A menudo, el texto se estructura en colecciones digitales de documentos cuya extensión (cantidad de caracteres) es variable, e.g. las páginas WEB o el contenido generado por los usuarios de redes sociales como Twitter. Para permitir el procesamiento de estas colecciones, primero se extrae el conjunto de palabras que aparece en ella y se ordena lexicográficamente; a este conjunto ordenado se le denomina vocabulario. Luego, el contenido de cada documento es representado algebraicamente por un vector, donde cada componente corresponderá a una de las palabras del vocabulario y cuyos valores estarán dados por las frecuencias de ocurrencia de cada palabra respectiva en el documento. Como consecuencia de la riqueza léxica del lenguaje, la cardinalidad del vocabulario es en general muy grande en comparación con los tamaños usados habitualmente por los algoritmos tradicionales de reconocimiento de patrones, i.e. partiendo del orden de los miles hasta los millones de palabras. Debido a esto, la tarea de agrupamiento automático de documentos tiene un alto costo de cómputo (tiempo de uso del procesador) y de almacenamiento (cantidad de memoria RAM y disco usado). Si a lo anterior se agrega que las cantidades de documentos en una colección puede superar los millones, entonces tanto las técnicas tradicionales de procesamiento, agrupamiento,  así como también las capacidades computacionales actuales  de una sola máquina resultan insuficientes o en el mejor de los casos los tiempos de respuesta son excesivos.
Often text data is arranged into digital collections of documents containing a variable number of  words. 
After extracting the terms appearing in every document an overall vocabulary that summarizes the words contained in the collection is built. 
In real text collections, this vocabulary may contain from tens of thousands to several hundred of thousands of words. 
In order to capture the lexical abundance of a language each document is computationally treated as a bag of words contained in the vocabulary, which in turn is represented as a vector whose dimensionality equals the size of the vocabulary.
This enormous size of the input data is far bigger than the size that traditional clustering algorithms can handle. 
Because of this, the task of automatic document clustering involves high computational and storage (RAM and secondary memory) costs. 
Along with this, when the number of documents is large (tens or hundreds of thousand and even millions of items)  traditional techniques for processing and clustering documents are insufficient under the current computational capabilities of a single machine and even high performance machines. 
Even in the most favorable scenario the storage and computational power tackle this challenge but the response time are excessive.

%Existe tres enfoques exitosos para la construcción de algoritmos de agrupamiento sobre grandes volúmenes de datos. El primero consiste en introducir restricciones en la cantidad de accesos a un documento (una sola pasada), el segundo en usar el paralelismo disponible en las arquitecturas multi-core actuales o usar GPUs (tarjetas gráficas) y por último, particionar el conjunto de datos en mútlples máquinas para su procesamiento distribuido.
There are three approaches that have been successfully applied to the construction of clustering algorithms capable of processing large volumes of data. The first one introduces constraints on the number of passes allowed on a document (related to the amount of time it is loaded into main memory). The second one exploits current multi-core architectures to perform parallel processing of the data. Both strategies do not tackle the problem on scenarios in which the data is distributed.
The last one combines the computational power of a single machine together with the scalable storage and processing capabilities of a distributed system in which data partitions are independently processed by several machines connected through a network.

\subsection*{Parallel processing architectures}

A multi-core processor (CPU) is a single chip that combines two or more independent processing units called \textit{cores}.
These processors are typically shared-memory architectures designed to support general purpose operations and contain from $2$ to several tens of cores optimized for sequential serial processing of a set of instructions.
A Graphical-Processing-Unit (GPU) is a massively parallel processor consisting of hundreds or even thousands of smaller, more efficient cores designed for handling multiple tasks simultaneously. At its beginning, GPUs were employed for hardware-assisted bitmap operations to assist in the display and usability of graphical operating systems. The release of the first GPU built with NVIDIA's CUDA Architecture enabled the usage of this hardware for general purporse computation~\citep{SK10}.
\vspace{15pt}
%In our opinion, an hybrid strategy in the usage of both parallel technologies is mandatory in order to address massive data scenarios. 
%A thumb rule to apply is to run on CPU small and non­parallelizable tasks and large and parallelizable tasks on GPU.


%Esta última línea de trabajo es promisoria dado que permite aprovechar las capacidades locales de computadores que no necesariamente deben ser servidores de gran escala con múltiples procesadores. Dentro de este enfoque existen dos contextos respecto del origen de los datos que deben ser diferenciados. Por una parte, existen problemas donde la estrategia usada para hacer frente al gran volúmen de datos consiste en distribuirlo entre distintos computadores, lo que conlleva un traslado de datos durante la ejecución del algoritmo \citep{N15}. Por otra parte, existe otro contexto en el que los datos se encuentran naturalmente distribuídos, e.g. colecciones documentales locales a distintas zonas geográficas, y donde además no es posible  centralizarlos, debido a costos de transmisión o por motivos de privacidad \citep{JW05,LHLX12}.
The last hybrid approach previously mentioned seems promising since it allows to exploit the local capabilities of single computers with multi-core architectures without sacrificing scalability to large data volumes because of its distributed design. Within this path there are two contexts regarding the data generation scenario. 
On the one hand, in some problems where the dataset is large but collected in a centralized fashion, the strategy employed consists in partitioning the collection into several machines or nodes of a network. 
This scheme leads to the transmission of a lot of data during the execution of the algorithm~\citep{N15}. On the other hand, there are some problems where the data is generated in a distributed fashion and it is not feasible to centralize the data because of high transmission costs or privacy issues~\citep{JW05,LHLX12}, e.g.\  search engines  work with document collections originated and stored in different geographical locations.



\section{Problem Definition}
%\section{Definición del Problema y Estado del Arte}
%Sea $X=\{X_1,X_2,\ldots ,X_p\}$ una colección de documentos donde \(\bigcup_{i=1}^p X_i=X\) y \(\forall i\neq j \in [1,\ldots ,p], X_i\cap X_j=\emptyset\). Esta colecci\'on se encuentra dividida en $p$ máquinas distinas, y a su vez, cada subconjunto $X_i$ se encuentra compuesto por $n_i$ vectores en $\mathbb{R}^d$. 
Let $X=\{X_1,X_2,\ldots ,X_p\}$ be a collection of documents where \(\bigcup_{i=1}^p X_i=X\) and \(\forall i\neq j \in [1,\ldots ,p], X_i\cap X_j=\emptyset\). This collection is partitioned into $p$ different machines of a network, and also each subset $X_i$ consists of $n_i$ vectors in $\mathbb{R}^d$. Additionally, in a distributed environment, data sites may be homogeneous or heterogeneous, depending if different sites contain data for exactly the same set of features or for different set of features respectively.  
%La tarea de Clustering distribuído consiste en obtener un conjunto de $k$ particiones \(C_1,C_2\ldots,C_k\) donde \(\bigcup_{i=1}^k C_i=X\) y además \(\forall i\neq j \in [1,\ldots ,k], C_i\cap C_j=\emptyset\), tal que cada una de ellas represente una o una parte de una categoría temática de la colección completa y a su vez todas las categorías esten representadas en los grupos o clusters $C_i$. Expresado de una manera distinta, esto significa que cada cluster contiene documentos más similares entre sí según su contenido que los que están en otros clusters.
The distributed data clustering task consists in obtaining $k$ data partitions \(C_1,C_2\ldots,C_k\) where \(\bigcup_{i=1}^k C_i=X\) and also \(\forall i\neq j \in [1,\ldots ,k], C_i\cap C_j=\emptyset\), such that each one of them represents a topical category of the overall collection and in turn all the categories are covered by the partitions or clusters $C_i$. Expressed in a different manner, this means that every cluster contains more similar documents according to their content in comparison to the other ones contained in different clusters.
%La similaridad entre vectores respresentates de documentos es usualmente medida mediante una función $S:\mathbb{R}^d\times \mathbb{R}^d\rightarrow[0,1]$ como por ejemplo la denominada similaridad del coseno:\[S(x_a,y_b)=\frac{<x_a,x_b>}{\|x_a||x_b|}\], donde $<u,v>$ con $u,v\in\mathbb{R}^d$ denota el producto interno entre dos vectores y $|u|$ es la norma euclideana del vector.
The similarity between document vectors is usually measured by means of a function $S:\mathbb{R}^d\times \mathbb{R}^d\rightarrow[0,1]$, e.g.\  the cosine similarity function: \[S(x_a,y_b)=\frac{<x_a,x_b>}{\|x_a||x_b|},\] where $<u,v>$ with $u,v\in\mathbb{R}^d$ denotes the inner product between two vectors and $|u|$ corresponds to the Euclidean norm of a vector.
%En pocas palabras, esta tarea consiste en encontrar una estructura de grupos compactos respecto a la similaridad entre sus miembros y homogéneos respecto de su contenido, de acuerdo a alguna medida de similitud como puede ser la del coseno.
In a nutshell, this task consists in finding a structure of compact and homogeneous groups with respect to a similarity value between their members that accounts for their content. 

%El modo general de operación de las técnicas de clustering sobre colecciones de datos distribuídas consiste de cuatro etapas: Inicialmente el conjunto de datos se encuentra particionado en varios nodos. Luego cada nodo genera un modelo de agrupamiento sobre su respectivo subconjunto local. Estos modelos son transmitidos a un nodo central (e.g. puede ser un conjunto de representantes identificados en cada grupo), el cual combina los modelos locales en una solución global. Opcionalmente, el modelo global puede ser transmitido a los otros nodos para que estos refinen sus modelos locales y se genere un nuevo modelo global depurado siguiendo las mismas etapas ya mencionadas. Este esquema se representa gráficamente en la figura
An overall operation mode of the distributed data clustering techniques consists of four stages: Initially, the dataset is partitioned into several nodes. Next, each node generates a clustering model over its corresponding local subset. These local models are then transmitted to a central node (e.g.\ A model can consists of representative points identified from a local clustering), which integrates the local models in a global solution. Eventually, this global model can be re-transmitted to the 
other nodes in order to refine their local models and then, by performing the same previous steps, a new depurated 
global model is built. This scheme is depicted in Figure~\ref{fig:distributed_clustering}.

\begin{figure}[!h]
\centering
\includegraphics[width=0.6\textwidth]{./figs/distributed_data_clustering_scheme_en.png}
%\caption{Esquema general del funcionamiento de las técnicas de clustering sobre datos distribuídos.}
\caption{Overall scheme of operation of the distributed data clustering techniques.}
\label{fig:distributed_clustering}
\end{figure}

\section{Parallel and Distributed approaches to High Dimensional Data Clustering}
%\subsection{Enfoques paralelos y distribuídos para el agrupamiento de datos masivos y de alta dimensionalidad}

\subsection{Main contributions on parallel clustering algorithms}

\citep{XJK99} presented a parallel version of \textit{DBSCAN} (\textit{PDBSCAN}). The authors presented the ‘shared-nothing’ architecture with multiple computers interconnected through a network. A fundamental component of a shared-nothing system is its distributed data structure. They introduced the dR*-tree, a distributed spatial index structure in which the data is spread among multiple computers and the indexes of the data are replicated on every computer. 
A performance evaluation showed that PDBSCAN offers nearly linear speed-up and has excellent scale-up and size-up behavior. 

In \citep{DM99}, the authors presented an algorithm that exploits the inherent data-parallelism in the K-Means algorithm. They analytically showed that the speed-up and the scale-up of the algorithm approached the optimal as the number of data points increased.

Another scalable approach based on secondary memory consisted in designing algorithms able to work  within the MapReduce framework. %\footnote{Hadoop MapReduce is a software solution that enables the construction of applications capable of processing large amounts of data (e.g. Terabytes) in a parallel fashion over big computer clusters.}. 
In this context we highlight the contribution made %by Das \textit{et al.} \citep{DDGR07} in which they propose an implementation of the EM algorithm and also 
by \citep{EIM11} in which they tackled the K-Median problem by using MapReduce.
%Das \textit{et al.} \citep{DDGR07} present an approach to filter recommendations for users of Google News in order to generate personalized recommendations in a collaborative way. They generate recommendations using three approaches: collaborative filtering using MinHash clustering, probabilistic Latent Semantic Indexing (pLSI), and co-visitation counts. The recommendations are combined from different algorithms using a linear model. The authors claim that the approach is content agnostic and consequently domain independent, making it easily adaptable for other applications and languages with minimal effort. 
%In \citep{EIM11} the authors design clustering algorithms that can be used in MapReduce, still one of the most popular programming environment for processing large data sets. They focus on the practical and popular clustering problems, k-center and k-median. 
The authors developed a fast clustering algorithm with constant factor approximation guarantees. The algorithm employed a sampling strategy to decrease the data size and after that a time consuming clustering algorithm such as local search or Lloyd's algorithm was performed on the resulting dataset. 
%The proposed algorithm had sufficient flexibility to be used in practice since they ran in a constant number of MapReduce rounds. The experiments showed that proposed algorithms solutions were similar to or better than the other algorithm solutions.

Another parallel approach for K-Means was presented by \citep{BMVKV12} and it is called K-Means$++$. The initialization of the K-Means algorithm is crucial, and the authors claimed that the proposed algorithm obtained an initial set of centers that is probably close to the optimum solution. A major downside of K-Means$++$ is its inherent sequential nature, which limits its applicability to massive data: one must make K passes over the data to find a good initial set of centers. In this work the
authors showed how to drastically reduce the number of passes needed to obtain, in parallel, a good initialization. This is unlike prevailing efforts on parallelizing K-Means that have mostly focused on the post-initialization phases of K-Means. The proposed initialization obtained a nearly optimal solution after a logarithmic number of passes, and then showed that in practice a constant number of passes suffices. 


\subsection{Clustering after applying dimensionality reduction}

\citep{KHSJ01} proposed a method to obtain the Principal Components (PCA) over heterogeneous and distributed data. Based on this contribution on dimensionality reduction they also proposed a clustering method that worked over high dimensional data. Once the global principal components were obtained by using the distributed method and transmitted to each node, local data was projected onto the components and then a traditional clustering technique was applied. Finally, a central node
integrated the local clusters in order to obtain a global clustering model.

\citep{LZO03} proposed an algorithm named the \textit{CoFD} algorithm, which was a non-distance based clustering algorithm for high dimensional spaces. Based on the Maximum Likelihood Principle, \textit{CoFD} attempted to optimize its parameter settings to maximize the likelihood between data points and the model generated by the parameters. The distributed version of the algorithm, called \textit{D-CoFD}, was also proposed. The authors claimed that the experimental results on both synthetic and real datasets showed the efficiency and effectiveness of \textit{CoFD} and \textit{D-CoFD} algorithms.

Several years later, \citep{LBK13} presented another algorithm for principal components extraction over distributed data. To this end, each node computed PCA over its local data and transmitted a fraction of them to a central node. This node used the received components to estimate the global principal components, which were later transmitted to each node. After this, in every node, local data were projected onto the global components and the projected data were used for computing a coreset by means of a distributed algorithm. The global coreset built from the local projected data was finally used to obtain a global clustering model.  

\citep{DDGN15} proposed a scalable algorithm that clusters hundreds of millions of WEB pages into hundreds of thousands of clusters. It does this on a single mid-range machine using efficient algorithms and compressed document representations (a binary signature for each document). It is applied to two WEB scale crawls covering tens of terabytes (ClueWeb09 and ClueWeb12). %contain 500 and 733 million web pages and were clustered into 500,000 to 700,000 clusters. Previous approaches clustered a sample that limits the maximum number of discoverable clusters. 
The proposed algorithm employed the entire collection for clustering and produced several orders of magnitude more clusters than the existing algorithms. Fine grained clustering is necessary for meaningful clustering in massive collections where the number of distinct topics grows linearly with collection size. Fine-grained clusters showed an improved cluster quality when assessed with two novel evaluations using ad-hoc search relevance judgments and Spam classifications for external validation. These evaluations solved the problem of assessing the quality of clusters where categorical labeling was unavailable.

\subsection{Distributed approaches for clustering based on density estimators}\label{sec:sa_dens}

\citep{KLM03} proposed a novel distributed clustering algorithm based on non-parametric kernel density estimation, which took into account the issues of privacy and communication costs in a low dimensional distributed data environment. %They state that several approaches to knowledge discovery and data mining, and in particular to clustering, have been developed, but only a few of them are designed for distributed data sources. 

\citep{JKP04} presented a scalable version of \textit{DBSCAN} that was also capable of operating over distributed collections. First, the best local representatives were selected depending on the number of points that each one represented and then those chosen points were sent to a central node. The central node clustered the local representatives into a single new model which was transmitted back to the other nodes to improve their local group structure. Unfortunately,
only experiments performed over low dimensional data were reported.

\subsection{Distributed approaches for clustering based on prototypes}\label{sec:sa_prot}

%Several works has been proposed over the years that are based on representative points. 
\citep{FZ00} described a technique to parallelize a family of center-based data clustering algorithms. The central idea of the proposed algorithm is to communicate only sufficient statistics, yielding linear speed-up with good efficiency. 
%The proposed technique did not involve approximation and may be used in conjunction with sampling or aggregation-based methods, such as BIRCH, to lessen the quality degradation of their approximation or to handle larger data sets. 
The authors demonstrated that even for relatively small problem sizes, it can be more cost~effective to cluster the data in~place using an exact distributed algorithm than to collect the data in one central location for clustering.

%\citep{ZLW08} propose an approximate K-Median clustering technique that works over streaming data, i.e. data is continuously collected. They propose a suite of algorithms for computing (1 + $\epsilon$ )-approximate k-median clustering over distributed data streams under three different topology settings: topology-oblivious, height-aware, and path-aware. The proposed algorithms reduce the maximum per node transmission to $polylog N$ (opposed to $\Omega(N)$ for transmitting the raw data). The authors perform simulations on a distributed stream system with both real and synthetic datasets composed of millions of data. In practice, the algorithms are able to reduce the data transmission to a small fraction of the original data. The results indicate that the algorithms are scalable with respect to the data volume, approximation factor, and the number of sites.

\citep{BEL13} provided novel extensions in a distributed scenario for two popular prototype-based strategies, namely k-median and K-Means. 
%These algorithms have provable guarantees and improve communication complexity over existing approaches. 
The proposed algorithms reduced the problem of finding a clustering with low cost to the problem of finding a coreset of small size. The authors provided a distributed method for constructing a global coreset which represented an interesting improve over the previous methods due to the reduction in terms of the communication complexity. Experimental results on large scale datasets showed that this approach is feasible and competitive against centralized techniques. 
%outperforms other coreset-based distributed clustering algorithms.

%\citep{NC14} tackle the problem of the parameter selection of the K-Means clustering algorithm by using evolutionary algorithms. Two different distribution approaches are adopted: the first obtains a final model identical to the centralized version of the clustering algorithm; the second generates and selects clusters for each distributed data subset and combines them afterwards. The algorithms are compared experimentally from two perspectives: the theoretical one, through asymptotic complexity analyses; and the experimental one, through a comparative evaluation of results obtained from a collection of experiments and statistical tests. The obtained results indicate which variant is more adequate for each application scenario.

\subsection{Distributed approaches for clustering based on parametric models}

\citep{MG03} presented a framework for clustering distributed data in unsupervised and semi~supervised scenarios, taking into account several requirements, such as privacy and communication costs. Instead of sharing portions of the original data, the proposed method transmitted to a central node the parameters of local generative models built at each site. They showed that the best representative of all the data is a certain "average" model, and empirically showed that this model
could be approximated quite well by generating artificial samples from the underlying distributions by using Markov Chain Monte Carlo techniques and then fitting a combined global model with a chosen parametric form to these samples. Also a new measure that quantifies privacy based on information theoretic concepts was proposed, and it showed that decreasing privacy leads to a higher quality of the combined model. They provided empirical results on different data types to highlight the
generality of the framework. The results showed that high quality distributed clustering can be achieved with little privacy loss and low communication cost.

\citep{KKPS05} proposed a distributed model-based clustering algorithm that used \textit{EM} for building local models based on mixtures of Gaussian distributions. They presented an efficient and effective algorithm for de\-ri\-ving and merging  local Gaussian distributions producing a meaningful global model. The experimental evaluation performed  demonstrated that the framework scaled-up in a highly distributed environment.

\subsection{Distributed approaches for Hierarchical clustering}

A hierarchical algorithm, called Collective Hierarchical Clustering (CHC), that works on distributed and heterogeneous data was presented by \citep{JK00}.  This algorithm first generated local cluster models and then combined them to generate the global cluster model of the data. The proposed algorithm ran in $O(|S|n^2)$ time, with a $O(|S|n)$ space requirement and $O(n)$ communication requirement, where $n$ is the number of elements in the dataset and $|S|$ is the number of data sites. 
This approach showed a significant improvement over naive methods with $O(n^2)$ communication costs in the case that the entire distance matrix is transmitted and $O(nm)$ communication costs to centralize the data, where m is the total number of features. A specific implementation based on the single link clustering and results comparing its performance with that of a centralized clustering algorithm were presented. Additionally, an analysis of the algorithm complexity, in terms of overall computation time and communication requirements, was presented.

\citep{JCHAC15} proposed a hierarchical clustering algorithm for distributed data that built the clusters by incrementally processing the data points. The authors re-stated the hierarchical clustering problem as a Minimum Spanning Tree construction problem over a graph. In order to integrate several local models they also proposed a technique for combining multiple Minimum Spanning Trees, assuming that these trees were obtained from disjoint subgraphs of the complete original graph. 
This mixture procedure iterated until a single tree was obtained, which corresponded to the hierarchical clustering originally pursued.

\subsection{Discussion}
%\subsection{Parallel and distributed approaches for massive and high dimensional data clustering}
%Hasta donde sabemos, gran parte de los esfuerzos en la literatura para la construcción de técnicas de agrupamiento capaces de operar en entornos donde los datos están distribuídos se ha enfocado en datos con datos cuya dimensionalidad es relativamente baja (menos 100 atributos) en comparación con las colecciones documentales (más de $10^4$ atributos). Sin embargo, a continuación detallamos los principales avances en el área de agrupamiento sobre datos distribuídos, destacando aquellas contribuciones dirigidas a los datos con alta dimensionalidad en sus representaciones computacionales.
There are several approaches in the literature for the construction of clustering techniques capable of operating in scenarios where the data is distributed. Unfortunately, most of them have been focused on the large number of data points but with low dimensional computational representations (less than 100 attributes) as in the cases of \citep{KLM03} and \citep{JKP04}. This contrasts to large document collections in which a document vector may have about $10^4$ attributes. For instance the RCV1 dataset processed by using the Scikit-learn framework\footnote{\url{http://scikit-learn.org/stable/datasets/rcv1.html}} contained $804414$ documents computationally represented as vectors spanned onto a space of $47236$ terms. Considering 4 Bytes to represent \textit{Float} values and without resorting to any sparse representation, the space required for loading this entire matrix in main memory is about $140$ Gigabytes which is beyond the RAM available in most computers.
%Following with this example, the number of non zero values of all these vectors is $60915113$ which aproximately corresponds to a $\%16$ of the total number of entries. Thus the space required for loading this entire matrix in main memory is $$ Bytes.

The increasing storage cost is not the only issue since as dimensionality grows traditional distance measures such as Cosine and Euclidean become more uniform, making clustering more difficult~\citep{ESK03}. Hierarchical and K-Medoids based algorithms, such as \citep{JK00} and \citep{JCHAC15}, are specially affected by this problem. Additionally, algorithms enhanced by the MapReduce framework, such as \citep{EIM11}, address the problem of processing a large number of documents but high dimensionality still affects each node of computation.

Some previous works deal with high dimensional data by using parallel approaches as in the case of \citep{DDGN15}. Besides the reduction in space requirements, the dataset is still limited by the available disk and RAM capacity of a single machine.

Considering the parallel processing improvements for centralized computation and the proposed strategies for distributed data processing made by previous contributions, the proposal presented in this project is related to two of the previously mentioned research trends, namely~\ref{sec:sa_dens} and~\ref{sec:sa_prot}.


\section{Proposal}
We propose a Distributed and Parallel strategy for dealing with high dimensional and massive text data clustering based on the Shared Nearest Neighbors scheme (SNN)~\citep{JP73}\citep{ESK03}. 
We expect that the inspiration in these two succesfully employed centralized approaches allows firstly, to build more robust algorithms for high dimensional data and secondly, to build scalable distributed algorithms based on representative points found in each node.

The proposed framework uses a Master/Slave model of computation in which a master node partitions the dataset and then each slave or worker node processes its assigned chunk of data in order to identify points that represents each group (\textit{Dataset partitioning} and \textit{Local model generation} stages in Figure
\ref{fig:distributed_clustering}). Then, these selected points are transmitted to the Master node in order to build an overall model which is then transmitted back to the workers to perform a model refining procedure~(\textit{Model refining} dotted loop in Figure \ref{fig:distributed_clustering}). This step can be performed several times, after which the group assignments are transmitted from each worker to the Master to generate the final result.

In addition to the distributed computation scheme, within each node it is possible to improve its performance by exploiting the underlying multicore technologies, e.g. in distance and k-nearest neighbor computation \citep{GMNAG15}. 
It is in this part of the proposed strategy where we expect that the introduction of parallel technologies such as GPU enhance the overall performance of the distributed algorithm.

The distributed computation strategy allows to decrease the cost of computing the complete dataset in one node which sometimes is unfeasible (Dataset sizes beyond the primary memory capacity). Furthermore, as each node can incorporate parallel routines (enabled by exploiting multicore processors or GPUs) to accelerate its partial computations the overall time cost of the algorithm is also improved.

%\section{Hipótesis}
\section{Hypothesis}
%Un enfoque distribuido del tipo maestro/esclavo de algoritmos de agrupamiento (clustering) locales y paralelos de una pasada, obtendrán resultados comparables a un esquema centralizado, en términos de medidas de desempeño de pureza y entropía, para problemas donde los datos son altamente dimensionales, masivos y naturalmente distribuidos provenientes de bases de datos documentales. 
%%A distributed master/slave approach for the design of single-pass and parallel clustering algorithms will enable the construction of methods capable of attaining results comparable to a centralized scheme, in terms of standard clustering performance measures such as Rand-score, Mutual Information and V-Measure, for high dimensional, massive and distributed document databases.

%* Distributed collections
%* Distributed and Parallel processing based on representative points
%* Shared Nearest Neighbors for representative point identification

A Master/Slave Clustering method along with a Shared-Nearest-Neighbor strategy for the identification of representative points in each worker, will outperform other state of the art distributed approaches (in terms of clustering quality measures such as V-measure and Rand-Index), over text collections in which documents are distributed across several nodes in a network and whose total size exceeds current RAM available in modern desktop computers.
\begin{itemize}
\item The computational procedure performed within each node will exploit modern parallel technologies existing in multicore-architectures and the usage of GPU accelerated computing.
\end{itemize}

%\section{Objetivos}
\section{Goals}
%El objetivo general de este trabajo consiste en desarrollar nuevas técnicas paralelas y distribuidas de Clustering para grandes volúmenes de documentos.
%En específico, este trabajo comprende los siguientes objetivos:
The general aim of this work consists in developing new parallel clustering techniques capable of dealing with large document databases that also can be stored in several nodes of a computer network.
Additionally, this work comprises the following specific objectives:
\begin{itemize}
%\item Desarrollar un esquema paralelo agrupamiento de documentos que realice sólo una pasada sobre la colección documental, i.e. no utilice memoria secundaria para procesar la colección de documentos.
\item To develop parallel clustering scheme for document collections that also performs a single-pass over each document, i.e.\  access to secondary memory during the algorithm execution is restricted.
%\item Extender el esquema paralelo propuesto para entornos donde la colección de documentos se encuentra distribuida en varias máquinas.
\item To extend the parallel scheme proposed to environments where the document collection is distributed into several machines.
%\item Validar los modelos propuestos tanto con datos de benchmark extraídos de sitios especializados, como con conjuntos de documentos reales (Noticias y textos de redes sociales).
\item To validate the proposed models with benchmark and real text data extracted from specialized sites such as UCI and NIST.
%\item Divulgar los resultados alcanzados en esta investigación en publicaciones registradas en el catalogo ISI, específicamente una publicación aceptada y otra enviada.
\item To disseminate the obtained results and methods in this research in scientific journals registered in the ISI catalog.
\end{itemize}


%\section{Metodología}
\section{Methodology}\label{sec:metodology}
%Para alcanzar las metas de esta propuesta, distinguimos las siguientes actividades generales y específicas:
In order to acomplish the goals of this project the following general and specific activities are distinguished:
\begin{enumerate}
%\item Estudio y Discusión de la Literatura relevante a la propuesta
%En el desarrollo de esta investigación se realizará una revisión completa y constante de la literatura relacionada con: 
    \item Constant revision and discussion of the state of the art literature related to:
        \begin{itemize}
            %\item Revisión de trabajos de Clustering, tanto paralelos y distribuidos, 
            \item Parallel and distributed Clustering methods.
            %\item Problemas teóricos y experimentales de agrupamiento de una pasada y paralelo sobre datos distribuidos. 
            \item Theoretical and experimental aspects involved in the single-pass and parallel clustering methods for distributed data. 
            %\item Revisión de trabajos relacionados con estrategias de Hashing para estimación de vecindarios ya sea centralizados o distribuidos, 
            \item Hashing strategies for the efficient estimation of neighborhood over centralized and distributed data collections.
            %\item Estudio de algoritmos de clustering para datos masivos y 
            \item Clustering methods for massive data collections.
            %\item Revisión de trabajos relacionados con Computación Distribuída y Paralela.
            \item Distributed and parallel Machine Learning algorithms.
        \end{itemize}
    %\item Diseño de los algoritmos y modelo propuestos. Basado en la hipótesis de la propuesta reconocemos los siguientes pasos en la formulación y diseño de los modelos de la propuesta:
    \item Design and construction of the proposed methods. In addition and considering the previously posed hypothesis and its validation, the following steps are established:
\begin{itemize}
%\item Modelamiento de ambientes donde existan datos distribuidos altamente dimensionales.
\item Modeling and analysis of results under environments presenting distributed and high dimensional data.

%\item Construcción de los modelos propuestos: Desarrollo de un enfoque de procesamiento distribuido para algoritmos de Clustering de una sola pasada capaces de operar en escenarios donde los conjuntos de datos son representados por vectores altamente dimensionales y se encuentran naturalmente distribuidos.
\item Construction of the proposed models. Development of distributed processing scheme for single-pass Clustering algorithm capable of operating under scenarios in which the vectors, each one representing a document, are distributed into several computers (workers).

%\item Estudio de como resumir o sintetizar volúmenes masivos de datos con el objetivo de generar representaciones computacionales económicas (e.g.\ coresets o minwise signatures).
\item An empirical study of techniques for summarizing massive text volumes (e.g. Coresets, medoids) with the aim to generate more succint snapshots of the current state of the stream. This is of particular interest in a distributed scenario in which a message passing based synchronization is performed between workers and the communication cost is restrictive. 

%\item Descripción de las propiedades teóricas de los modelos: Durante esta fase nos enfocaremos en la generación de conocimiento centralizado a partir de modelos locales de Clustering.
\item Analysis of the algorithmic complexity in terms of space and time of the proposed models.
\end{itemize}
%\item Implementación y Optimización de los Algoritmos Propuestos
%Los algoritmos propuestos serán implementados en un nuevo lenguaje llamado JULIA (http://www.julialang.org). Desarrollaremos una estrategia compuesta de módulos capaces de ser evaluados e intercambiados independientemente. Durante esta fase de desarrollo, identificamos los siguientes pasos (los cuales pueden ser iterados): a) Construcción de una plataforma general para construir la arquitectura de los prototipos, b) Implementación de diversos escenarios distribuidos, c) Implementación de algoritmos del estado del arte con fines comparativos, d) Implementación de las técnicas de validación, e) Elaboración de soporte para clusters y tecnologías de Cloud Computing tales como MPI, OpenStack y Amazon EC2, con el fin de simular el problema en un ambiente real.
\item During the development of the algorithm implementations we identified the following steps: (a) Construction of a modular framework adequately general to serve as a basis for the posterior construction of the proposal and its variants\. (b) The implementation and testing of several distributed scenarios (e.g. Balanced and Imbalance load in workers, arrival of new instances and its posterior labeling and the incorporation of concept-drift by introducing new classes to the algorithm).
(c) Implementation and evaluation of baseline algorithms of the state of the art (e.g. Spark K-Means, Spark-DBSCAN and RACHET [\citep{SOGM02}]).%TODO: AGREGAR RACHET AL ESTADO DEL ARTE

%\item Diseño de los experimentos y validación de los algoritmos propuestos.
%Para validar los modelos propuestos, tenemos planificado usar tanto datos “benchmark” conocidos en el área, como datos provenientes de problemas reales de interés regional. Los datos de “benchmark” serán recolectados de sitios web de acceso publico del área de Máquinas de Aprendizaje. En la mayoría de los casos, los conjuntos de datos tienen asociada a cada documento una etiqueta que indica su verdadera clase. A las medidas de evaluación que comparan las etiquetas asignadas por el método de clustering con las verdaderas se les denomina medidas externas. Las medidas externas que usaremos con el fin de validar la propuesta son: Rand Score, Mutual Information, Homogeneity, Completeness y V-Measure. Para aquellos conjuntos de datos que no cuenten con etiquetas verdaderas, se usará la medida interna Silhouette \citep{R87}, la cual cuantifica que tan bien diferenciados están los grupos encontrados por el método bajo evaluación.
%La etapa de validación será realizada mediante la corrida de 20 experimentos con particionamientos aleatorios para cada uno de los conjuntos de datos utilizados. Los métodos propuestos serán comparados con modelos del estado del arte en al menos 10 conjuntos de datos, tanto reales como sintéticos obtenidos de dos fuentes:
\item Design of the experiments and validation of the proposed algorithms. 
    In order to validate the proposed techniques, we plan to use benchmark datasets coming from several sources; namely, the UCI machine learning repository\footnote{\url{http://archive.ics.uci.edu/ml/index.php}}, the Tipster text collection\footnote{\url{https://catalog.ldc.upenn.edu/LDC93T3A}} and the GOV2 text collection\footnote{\url{http://ir.dcs.gla.ac.uk/test_collections/gov2-summary.htm}}. 
In some cases the document set is labeled or at least some labels can be inferred from relevance judges (e.g.\ the Tipster collection), and in some other no label information is available (e.g. GOV2 collection).
 There exist several measures for the quantification of clustering quality. When labeling information is available external measures such as V-measure [\citep{RH07}] and Adjusted Rand Index [\citep{HA85}] are applied. 
In the absence of this information, internal measures such as the Silhouette [\citep{R87}] coefficient and CVNN [\citep{LLXGWW13}] are utilized.
As the partitioning of the data is performed at random, the validation will be conducted by executing 20 runs over each dataset. Then, the average and standard deviation of each attained quality measure will be reported.
    \item Colaboration and dissemination activities:
        \begin{itemize}
            %\item Se realizará un Seminario sobre técnicas de clustering y aplicaciones sobre problemas específicos de aprendizaje distribuido, Big Data, métodos de ensamblado, con colegas y estudiantes. Este seminario incluirá discusiones con alumnos de pre y post grado.
            \item Seminar with topics about Clustering and Non-Supervised methods and their applications over problems involving  distributed and massive data collections. Researchers, practitioners and students (undergraduate and postgraduate) will be included. 
            %\item Participación en conferencias nacionales e internacionales relacionadas con máquinas de aprendizaje,  sistemas inteligentes distribuidos y descubrimiento de conocimiento desde grandes bases de datos.
            \item Participation in national and international conferences related to Machine Learning techniques, Intelligent Systems for distributed data and Knowledge extraction from large databases (at least one per year).
            %\item Se mantendrá contacto con especialistas internacionales de temas de investigación relacionados, ya sea tanto del medio nacional, latinoamericano, como de Europa y América del norte.
            \item Establishing contact for cooperation with related research groups either in the national as in the international sphere.
        \end{itemize}
\end{enumerate}

\section{Work Plan}
%\section{Plan de Trabajo}
%MODIFICAR (HACER CALZAR CON OBJETIVOS)
The stages were explained in the Methodology section and are the following:
%Las etapas fueron descritas en el punto~\ref{sec:metodology} y son las siguientes:
\begin{itemize}
\item \textit{Stage 1}: %Estudio y discusión de la literatura relevante para la propuesta.
Study and Discussion of the relevant literature for the proposal.
\item \textit{Stage 2}: %Dise\~no de los modelos y algoritmos propuestos:
Design of the proposed models and algorithms: 
\begin{enumerate}
\item %Modelamiento de ambientes con patrones distribuidos.
Design of a processing strategy for distributed data clustering.
\item %Construcción de los modelos propuestos.
Implementation of the proposed scheme.
%\item %Descripción teórica de las capacidades y propiedades de los modelos.
%Theoretical discussion about the scalability of the proposal in terms of algorithmic complexity (time and space).
%(3.1) Modelling environments with distributed patterns, with hybrid fragmentation. (3.2) Modelling non-stationary environments with large distributed sources, with the aforementioned fragmentation type. (3.3) Construction of the proposed models. (3.4) Description of the theoretical capabilities and properties of the models.
\end{enumerate}
\item \textit{Stage 3}: %Dise\~no de experimentos y estrategia de validación de los algoritmos propuestos.
    Design of Experiments and validation of the proposed algorithms: 
\begin{enumerate}
\item %Generación de conjuntos de datos sintéticos.
    Synthetic data generation.
\item %Recolección de datos reales y sintéticos.
    Processing real text data. For the evaluation of clustering algorithms a general practice consists in using datasets employed for classification since these have labeling information, hence external evaluation measures for Clustering can be used. Additionally, some datasets not labeled but large in size, such as GOV2, will be considered. 
    %\item Procedimiento de validación.
\item %Análisis comparativo.
    Comparative analysis against state of the art techniques. External and internal measures will be used and the achieved values will be compared against the ones attained by techniques such as DBSCAN and K-Means implemented within the MapReduce framework for distributed computing. 

%(5.1) Generating Synthetic datasets. (5.2) Recollecting synthetic and real datasets. (5.3) Validation Procedure. (5.4) Comparative Analysis.
\end{enumerate}
\item \textit{Stage 4}: %Diseminación de los resultados obtenidos en este proyecto.
    Dissemination of the attained results.
\item \textit{Stage 5}: Study of algorithmic complexity in terms of time and space.
    \begin{itemize}
        \item Exploration of potential optimizations of the proposed algorithms.
        \item Discussion about the scalability of the proposed algorithms.
    \end{itemize}
\item \textit{Stage 6}: Integration of the proposals with GPU computation at the Master and Workers levels. In this stage, GPU will be specially used to improve array products for distance computation.
\item \textit{Stage 7}: %Diseminación de los resultados obtenidos en este proyecto.
    Final dissemination of the attained results and contributions of this project.
\end{itemize}

%(PROY HECTOR)During the year 2017 , we will define the Distributed Learning model with hybrid data fragmentation from a statistical learning point of view (S1), we will concentrate efforts on the construction of a meta-ensemble model for distributed learning with hybrid fragmentation  (S2), we will propose mechanisms to quantitatively describe disimilarities between distributed data sources (S4), we will study ensemble models in order to fulfill the problems requirement (S2) and we will empirically evaluate and compare the proposed algorithms (S3).\\

%Durante el a\~no 2017 definiremos el modelo de clustering distribuido desde el punto de vista de la extracción de representantes desde cada fuente que permitan una posterior identificación de los grupos subyacentes a toda la colección (Etapa 1). Concentraremos esfuerzos en el estudio y construcción de técnicas basadas en Coresets y en distancias derivadas de vecinos más cercanos compartidos (Etapa 2 y 3). Por último, evaluaremos empíricamente y compararemos los algoritmos propuestos (Etapa 4).

In summary, during year 2018, we will concentrate on the development of a distributed framework that addresses the high dimensional data clustering task from a core-representatives approach. That is, the overall data clustering is obtained from representatives chosen independently in each local worker (S1). Therefore the effort will be put on techniques based on Coresets and distances derived from Nearest-Neighbors shared between data points (S2).
In order to assess the performance of the proposal, real high dimensional datasets will be used and a thorough experimental validation will be made (measuring clustering quality through several executions of the algorithm and under different parameter settings) (S3).
At the end of 2018, the built algorithms together with their attained results will be reported in at least one indexed journal (S4).



%(PROY HECTOR)During the year 2018 we will concentrate efforts on the development of theoretical criteria and algorithms to handle distributed scenarios, with different underlying laws of probability (S3, S4), in scenarios where temporal concept drift is present (S5). We will propose a model that is able to have a good performance in the presence of all these scenarios. Validation with synthetic data sets will be performed (S6).\\

%Durante el a\~no 2018 concentraremos nuestros esfuerzos en el estudio teórico de propiedades de los algoritmos que permitan justificar su desempeño e identificar escenarios menos favorables para su operación. Este análisis facilitará la comprensión de la contribución de los algoritmos propuestos para su posterior diseminación (Etapa 5).

During the year 2019, we will focus firstly, on the study of the algorithmic complexity of the overall distributed framework along with the complexity of the tasks performed by each worker (S5). 
Lastly, the effort will be put on the usage of GPU computation to enhance performance times specially on the computation of distances and neighborhoods (these are the two most expensive operations) (S6). 
The improvements and findings obtained during this year will be published in at least one indexed
journal (S7).


\section{Work in progress}
% El esquema anterior fue pensado en un escenario centralizado en donde es posible calcular las distancias entre todos los pares de puntos. Para un conjunto de datos con $n$ puntos, cada uno representado por un vector en $\mathbb{R}^d$, el costo de calcular estas distancias es O($n^2 * d$) y de almacenarlas es O($n^2$). Al considerar grandes volúmenes de datos representados además por vectores altamente dimensionales, resulta claro que el desempe\~no del método difícilmente escale.

The Shared-Nearest-Neighbor clustering algorithm presented by~\citep{ESK03} proposes the integration of the DBSCAN clustering algorithm along with a shared-neighbors based similarity measure. 
This scheme tackles the variable density, size among clusters and also the curse of dimensionality in a centralized setting. 
In spite of its valuable contribution this method assumes that pairwise distance/similarity computation and storage is always feasible. 
Due to the quadratic time complexity involved in the computation of the distances and the similar space complexity to store these data, centralized methods that rely on this calculations do not scale in massive data scenarios.  

%\section{Trabajo en Progreso}
%Una enfoque inicial de exploración para el dise\~no y construcción de algoritmos de clustering de colecciones documentales distribuidas consiste en una adaptación del algoritmo de \citep{ESK03} para un contexto distribuído. 
%En este esquema se ataca primero la alta dimensionalidad de los vectores mediante medidas de distancia basadas en la cantidad de vecinos que comparten dos puntos cualquiera. 
%Por otra parte, la gran cantidad de datos y el ruído se ataca mediante la selección de representantes, denominados \textit{Core-points}. 
%Para esto se utilizan dos parámetros, \textbf{Eps} y \textbf{MinPts}. 
%Luego, un punto será representante solamente si comparte más de \textbf{Eps} vecinos con más de \textbf{MinPts} puntos. 
%Finalmente, se etiquetan con el mismo número de grupo aquellos \textit{Core-points} que se compartan más de \textbf{Eps} puntos, y luego el resto de los puntos recibe la etiqueta de su \textit{Core-point} más cercano en término de vecinos cercanos compartidos. 
%En el caso de que el \textit{Core-point} más cercano esté a distancia menor que \textbf{Eps}, se identifica como ruído.


An initial approach that we took is to explore the construction of clustering algorithms capable of dealing with high dimensional and distributed text collections by adapting the centralized algorithm proposed by~\citep{ESK03}.
Consequently, the curse of dimensionality is addressed by treating the distance between two data points as a function of the nearest neighbors they share (nearest in terms of a traditional metric such as the euclidean distance).
Additionally, the large number of observations and the potential noise or abnormal data points are addressed by a representatives selection mechanism performed on each worker.
Is in this way that each worker reports only those points from its local subset that share a sufficient quantity of neighbors with their nearest neighbors. 
The rationale behind this scheme is that the presence of noise is denoted by more or less isolated data points.
Then, the master node concentrates all these core points, builds a network based on their shared-nearest-neighbor similarity and labels them by using a network propagation scheme such as the one proposed by~\citep{RAK07}.
Finally, those core points along with their labels are transmitted back to the worker nodes, each worker labels its local subset by using the label of its nearest corepoint and then retransmits this information to the master node. Once the master node receives all the labels it reports the final clustering of the collection.


%El trabajo que actualmente estamos realizando considera que la colección se encuentra particionada aleatoriamente en un conjunto de nodos o máquinas. En cada uno de estos nodos se usa el esquema de \citep{ESK03} para identificar \textit{Core-points} y etiquetar los datos locales. Tomando algunas ideas propuestas para la construcción de core-sets distribuídos por \citep{BEL13}, se realiza en cada máquina una selección de \textit{Core-points} aleatoria con pesos inversamente proporcionales a la cantidad de puntos con la etiqueta del \textit{Core-point} y luego se transmiten los puntos seleccionados aun nodo central. Es importante mencionar que, a diferencia del trabajo en curso que se describe en esta sección, las contribuciones en esta línea no consideran en general datos de alta dimensionalidad (vectores con miles de atributos) ni tampoco grupos que pueden tener formas no esféricas. De esta manera, serán seleccionados con mayor probabilidad aquellos \textit{Core-points} ubicados en grupos más pequeños, buscando así representar a todos los grupos independientemente de que tan peque\~nos sean. La cantidad de puntos seleccionados es un parámetro expresado en términos de porcentaje y es precisamente este mecanismo el que permite atacar los problemas generados por los grandes volúmenes de datos. Así en el nodo central se realizará nuevamente un clustering usando la medida de distancia basada en vecinos más cercanos compartidos, pero únicamente sobre una fracción de todos los \textit{Core-points}. El agrupamiento final contendrá los grupos de la colección completa, lo que corresponderá al resumen esperado de la colección.

%Actualmente, hemos realizado experimentos sobre datos sintéticos con formas esfericas y no esfericas, obteniendo resultados interesantes. Para poder realizar afirmaciones concluyentes, aplicaremos el método sobre colecciones documentales reales.

The ongoing work described above has been tested on synthetic and real text data but with reduced size. An accepted article at the \textit{CIARP-2017}\footnote{www.ciarp2017.org} conference shows some preliminary results attained with the proposed scheme. The article is titled \textit{``A Distributed Shared Nearest Neighbors Clustering Algorithm''} and is authored by the principal researcher, the sponsoring researcher and a coleague from the \textit{Universidad T\'ecnica Federico Santa Mar\'ia}.

We expect to be able to test the algorithm with really large collections (more than several hundred of thousands of documents) and also to improve the local workers by exploiting the power of GPU computing. 


\section{Available Resources}
%\section{Recursos disponibles}
%Los recursos disponibles para este proyecto en la Escuela de Ingeniería Informática de la Pontificia Universidad Católica de Valparaíso son:
The resources available for this project at the \textit{Escuela de Ingeniería Informática} of the \textit{Pontificia Universidad Católica de Valparaíso} are:

\begin{itemize}
%\item Biblioteca de la Universidad y biblioteca personal del investigador patrocinante.
    \item University library, free access to electronic libraries such as IEEE and Springer and the personal library of the sponsoring researcher at the previously mentioned campus.
    \item A room for conferences.
    \item An office with telephone and fast Internet connection.
    \item A Dell R730 server (20 physical cores and 128 GB RAM).
\end{itemize}

Open access to programming languages such as Julia, Python, Java, C++ and R, and to the Latex document preparation system.
%Existen varias licencias de software gratuitas en internet, como por ejemplo: Julia, Python, Java, R, Latex, etc. 

\newpage
\bibliographystyle{apalike}
% \bibliography{references}
\begin{thebibliography}{}

\bibitem[Bahmani et~al., 2012]{BMVKV12}
Bahmani, B., Moseley, B., Vattani, A., Kumar, R., and Vassilvitskii, S. (2012).
\newblock{Scalable K-Means++}.
\newblock{\em Proceedings of the VLDB Endowment (PVLDB)}, 5:622--633.

\bibitem[Balcan et~al., 2013]{BEL13}
Balcan, M.~F., Ehrlich, S., and Liang, Y. (2013).
\newblock{Distributed K-Means and K-Median Clustering on General
  Topologies}.
\newblock{\em Advances in Neural Information Processing Systems 26 (NIPS
  2013)}, pages 1--9.

\bibitem[Crestani and Markov, 2013]{CM13}
Crestani, F., and Markov, I. (2013).
\newblock{Distributed Information Retrieval and Applications}.
\newblock{\em 35th European Conference on IR Research}, 865--868.

\bibitem[Das et~al., 2007]{DDGR07}
Das, A., Datar, M., Garg, A., and Rajaram, S. (2007).
\newblock Google news personalization: scalable online collaborative filtering.
\newblock In {\em Proceedings of the 16th international conference on World
  Wide Web}, pages 271--280. ACM.

\bibitem[De~Vries et~al., 2015]{DDGN15}
De~Vries, C.~M., De~Vine, L., Geva, S., and Nayak, R. (2015).
\newblock Parallel streaming signature em-tree: A clustering algorithm for web
  scale applications.
\newblock In {\em Proceedings of the 24th International Conference on World
  Wide Web}, pages 216--226. ACM.


\bibitem[De~Vries et~al., 2009]{DG09}
De~Vries, C.~M. and Geva, S. (2009).
\newblock {K~Tree: large scale document clustering}.
\newblock {\em Proceedings of the 32nd international ACM SIGIR conference on Research and development in Information Retrieval}, pp. 718--719, ACM.

\bibitem[Dhillon and Modha, 1999]{DM99}
Dhillon, I.~S. and Modha, D.~S. (1999).
\newblock {A data-clustering algorithm on distributed memory multiprocessors}.
\newblock {\em LargeScale Parallel Data Mining}, 1759(802):245--260.

\bibitem[Ene et~al., 2011]{EIM11}
Ene, A., Im, S., Moseley, B. (2011).
\newblock {Fast Clustering using MapReduce}.
\newblock {\em Kdd}, 681--689.

\bibitem[Ert{\"o}z et~al., 2003]{ESK03}
Ert{\"o}z, L., Steinbach, M., and Kumar, V. (2003).
\newblock {Finding clusters of different sizes, shapes, and densities in noisy, high dimensional data}.
\newblock {\em Proceedings of the SIAM International Conference on Data Mining}, 47--58.

% \bibitem[Ester et~al., 1996]{EKSX96}
% Ester, M., Kriegel, H., Sander, J., and Xu, X. (1996)
% \newblock {A density-based algorithm for discovering clusters in large spatial databases with noise}.
% \newblock {\em Proceedings of the Second International Conference on Knowledge Discovery and Data Mining}, 96:226--231.

\bibitem[Forman and Zhang, 2000]{FZ00}
Forman, G. and Zhang, B. (2000).
\newblock {Distributed data clustering can be efficient and exact}.
\newblock {\em ACM SIGKDD Explorations Newsletter}, 2(2):34--38.


\bibitem[Gavahi et~al., 2015]{GMNAG15}
Gavahi, M., Mirzaei, R., Nazarbeygi, A., Ahmadzadeh, A. and Gorgin, S. (2015).
\newblock{High performance GPU implementation of k-NN based on Mahalanobis distance}.
\newblock{\em International Symposium on Computer Science and Software Engineering},  1--6.


\bibitem[Han et~al., 2011]{HK11}
Han, J., Pei, J., and Kamber, M. (2011).
\newblock{\em Data mining: concepts and techniques}.
\newblock Elsevier.

\bibitem[Hubert and Arabie, 1985]{HA85}
Hubert, L. and Arabie, P. (1985).
\newblock{Comparing partitions}.
\newblock{\em Journal of Classification}, 2:1, 193--218, Springer.

\bibitem[Jagannathan et~al., 2005]{JW05}
Jagannathan, G., and Wright, R. N. (2005).
\newblock{Privacy-preserving Distributed K-Means Clustering over Arbitrarily Partitioned Data}.
\newblock{\em Proceedings of the Eleventh ACM SIGKDD International Conference on Knowledge Discovery in Data Mining}, 593--599.

\bibitem[Januzaj et~al., 2003]{JKP03}
Januzaj, E., Kriegel, H.-P., and Pfeifle, M. (2003).
\newblock{Towards Effective and Efficient Distributed Clustering}.
\newblock{\em Workshop on Clustering Large Data Sets}, pages 49--58.

\bibitem[Januzaj et~al., 2004]{JKP04}
Januzaj, E., Kriegel, H.-P., and Pfeifle, M. (2004).
\newblock{Scalable Density-Based Distributed Clustering}.
\newblock pages 231--244.

\bibitem[Jarvis and Patrick, 1973]{JP73}
Jarvis, R.A. and Patrick, E.A. (1973).
\newblock{Clustering Using a Similarity Measure Based on Shared Near Neighbors}.
\newblock{\em IEEE Transactions on Computers}, 22:1025--1034.


\bibitem[Jin et~al., 2015]{JCHAC15}
Jin, C., Chen, Z., Hendrix, W., Agrawal, A., and Choudhary, A. (2015).
\newblock{Incremental, Distributed Single-linkage Hierarchical Clustering
  Algorithm Using Mapreduce}.
\newblock{\em Proceedings of the Symposium on High Performance Computing},
  pages 83--92.

\bibitem[Johnson and Kargupta, 2000]{JK00}
Johnson, E. and Kargupta, H. (2000).
\newblock{Collective, hierarchical clustering from distributed, heterogeneous
  data}.
\newblock{\em Lecture Notes in Computer Science}, 1759:221--244.

\bibitem[Kargupta et~al., 2001]{KHSJ01}
Kargupta, H., Huang, W., Sivakumar, K., and Johnson, E. (2001).
\newblock{Distributed clustering using collective principal component analysis}.
\newblock{\em Knowledge and Information Systems}, 3(4):422--448.

\bibitem[Klusch et~al., 2003]{KLM03}
Klusch, M., Lodi, S., and Moro, G. (2003).
\newblock{Distributed clustering based on sampling local density estimates}.
\newblock{\em IJCAI International Joint Conference on Artificial
  Intelligence}, pages 485--490.

\bibitem[Kriegel et~al., 2005]{KKPS05}
Kriegel, H.-p., Kr, P., Pryakhin, A., and Schubert, M. (2005).
\newblock{Effective and Efficient Distributed Model-based Clustering}.

\bibitem[Li et~al., 2003]{LZO03}
Li, T., Zhu, S., and Ogihara, M. (2003).
\newblock{Algorithms for Clustering High Dimensional and Distributed Data}.
\newblock{\em Intelligent Data Analysis Journal}, 7(February):1--36.

\bibitem[Liang et~al., 2013]{LBK13}
Liang, Y., Balcan, M.-f., and Kanchanapally, V. (2013).
\newblock{Distributed PCA and K-Means Clustering}.
\newblock{\em The Big Learning Workshop in NIPS 2013}, pages 1--8.

\bibitem[Liu et~al., 2012]{LHLX12}
Liu, J., Huang, J. Z., Luo, J., and Xiong, L. (2012).
\newblock{Privacy Preserving Distributed DBSCAN Clustering}.
\newblock{\em Proceedings of the 2012 Joint EDBT/ICDT Workshops}, 177--185.

\bibitem[Liu et~al., 2013]{LLXGWW13}
Liu, Y., Li, Z. Xiong, H., Gao, X., Wu, J. and Wu, S. (2013).
\newblock{Understanding and Enhancement of Internal Clustering Validation Measures}. 
\newblock{\em IEEE Transactions on Cybernetics}, 43:3, pp. 982--994, June 2013.

\bibitem[Merugu and Ghosh, 2003]{MG03}
Merugu, S. and Ghosh, J. (2003).
\newblock{Privacy-preserving Distributed Clustering using Generative Models}.
\newblock{\em Proceedings of the 3rd IEEE International Conference on Data
  Mining (ICDM)}, pp. 0--7.

\bibitem[Nagwani, 2015]{N15}
Nagwani, N. K. (2015).
\newblock{Summarizing large text collection using topic modeling and clustering based on MapReduce framework}.
\newblock{\em Journal of Big Data}, 2:1--18.

\bibitem[Naldi and Campello, 2014]{NC14}
Naldi, M.~C. and Campello, R. J. G.~B. (2014).
\newblock{Evolutionary K-Means for distributed data sets}.
\newblock{\em Neurocomputing}, 127:30--42.

\bibitem[Nayak et~al., 2014]{NMDG14}
Nayak, R. and Mills, R. and De~Vries, C. and Geva, S. (2014).
\newblock{Clustering and Labeling a Web Scale Document Collection using Wikipedia clusters}
\newblock{\em Proceedings of the 5th International Workshop on Web-scale Knowledge Representation Retrieval}, pp. 23--30, ACM.


\bibitem[Raghavan et~al., 2007]{RAK07}
Raghavan, U., N., Albert, R. and Kumara, S. (2007).
\newblock{Near linear time algorithm to detect community structures in large-scale networks}.
\newblock{\em Physical review}, 76(3).

\bibitem[Rosenberg and Hirschberg, 2007]{RH07}
Rosenberg, A. and Hirschberg, J. (2007).
\newblock{V-Measure: A Conditional Entropy-Based External Cluster Evaluation Measure.}.
\newblock{\em Proceedings of Conference on Empirical Methods in Natural Language Processing and Computational Natural Language Learning (EMNLP-CoNLL)}, 7:410--420.


\bibitem[Rousseeuw, 1987]{R87}
Rousseeuw Peter J. (1987). 
\newblock{Silhouettes: A graphical aid to the interpretation and validation of cluster analysis}.
\newblock{\em Journal of Computational and Applied Mathematics}, 20:53--65.

\bibitem[Samatova et~al., 2002]{SOGM02}
Samatova, N. F. and Ostrouchov, G. and Geist, A. and Melechko, A. V. (2002).
\newblock{RACHET: An Efficient Cover-Based Merging of Clustering Hierarchies from Distributed Datasets}.
\newblock{Journal of Distributed and Parallel Databases}, 11:2, 157--180, Springer.


\bibitem[Sanders and Kandrot, 2010]{SK10}
Sanders, J. and Kandrot, E. (2010).
\newblock{CUDA by Example: An Introduction to General-Purpose GPU Programming} (1st edition).
\newblock{Addison-Wesley Professional}.


\bibitem[Xu et~al., 2002]{XJK99}
Xu, X., J{\"{a}}ger, J., and Kriegel, H. (2002).
\newblock{A fast parallel clustering algorithm for large spatial databases}.
\newblock{\em High Performance Data Mining}, 290:263--290.

\bibitem[Qi et~al., 2008]{ZLW08}
Qi, Z., Jinze, L., and Wei, W. (2008).
\newblock{Approximate clustering on distributed data streams}.
\newblock{\em Proceedings - International Conference on Data Engineering},
  00:1131--1139.


\end{thebibliography}

\end{document}
